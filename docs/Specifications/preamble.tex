\documentclass[12pt]{scrreprt}
\usepackage[a4paper, margin=2.5cm]{geometry}
\usepackage{xcolor}
\usepackage{listings}
\usepackage{underscore}
\usepackage[english]{babel}
\usepackage{scrlayer-scrpage}
\usepackage{setspace}
\usepackage{titlesec}
\usepackage[utf8]{inputenc}
\usepackage{placeins}
\usepackage{float}
\usepackage{graphicx}
\usepackage{caption}    % 允许子图使用 caption
\usepackage{subcaption} 
\usepackage{array}
\usepackage{colortbl}
\usepackage{booktabs}
\usepackage[bookmarks=true]{hyperref}
\usepackage{parskip}           % 段落间距
\setlength{\parskip}{0.5em}    % 紧凑段落间距
\renewcommand{\baselinestretch}{1.05} % 行距微调

% main.tex导言区
\usepackage{titlesec}
\titleformat{\chapter}[block]
{\normalfont\huge\bfseries}{\thechapter}{1em}{}
\titleformat{\section}[block]
{\normalfont\Large\bfseries}{\thesection}{1em}{}
\titleformat{\subsection}[block]
{\normalfont\large\bfseries}{\thesubsection}{1em}{}

% 保持编号与标题的经典间距
\makeatletter
\renewcommand{\@seccntformat}[1]{\csname the#1\endcsname\quad}
\makeatother

% 现代间距系统(单位:pt)
\titlespacing*{\chapter}{0pt}{25pt}{15pt}  % 上间距25pt/下间距15pt
\titlespacing*{\section}{0pt}{15pt}{7pt}
\titlespacing*{\subsection}{0pt}{10pt}{5pt}

% 确保子章节编号
\setcounter{secnumdepth}{3}
\setcounter{tocdepth}{3} % 目录显示到三级标题

% 封面字体优化
\usepackage{anyfontsize}        % 精确字体控制
\usepackage[sfdefault]{noto}   % 现代无衬线字体(需要安装Noto Sans字体)

%refers
\usepackage[style=ieee, sorting=none, backend=biber]{biblatex}
\addbibresource{re.bib}

% Hyperref配置
\hypersetup{
    pdftitle={Software Requirement Specification},
    pdfauthor={Jean-Philippe Eisenbarth},
    pdfsubject={TeX and LaTeX},
    pdfkeywords={TeX, LaTeX, SRS, Requirements},
    colorlinks=true,
    linkcolor=blue!70!black,
    citecolor=green!60!black,
    filecolor=magenta,
    urlcolor=cyan!70!black,
    linktoc=page
}

% 全局格式设置
\setstretch{1.1}
\titleformat{\chapter}[display]
{\normalfont\huge\bfseries}{\chaptertitlename\ \thechapter}{20pt}{\Huge}
\titlespacing*{\chapter}{0pt}{-30pt}{40pt}

% 自定义命令
\newcommand{\placeholder}[1]{\textcolor{gray!70}{\textlangle\,\textit{#1}\,\textrangle}}
\newcommand{\req}[2]{\item[REQ-#1:] #2}